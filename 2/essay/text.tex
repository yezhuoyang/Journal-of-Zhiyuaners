

\section{正文}
各位同学大家好,我是致远物理13级焦小沛,我目前在清华大学高等研究院读直博二年级。下面就目前我的研究方向为大家做个简要的介绍。目前我的研究方向是计算系统生物学,属于应用数学的范畴。该研究方向主要关注生物体内的各种系统过程,并利用数学进行定量的建模,进行合理的解释以及预测。通过对该具体生物问题的研究,进一步的进行定量刻画,甚至反过来启发我们从中发现有趣的数学现象,推动新的数学方法的发展。

下面具体到我个人的课题,我主要研究癌症的发展过程中的数学建模,癌症的发生和发展属于复杂系统,具有多尺度,多层次以及巨大的自由度等特点。该复杂系统主要分为基因表达层面,单细胞层面和多细胞层面。在基因表达层面,我们将考虑一个复杂的基因调控网络,该网络中每个结点代表一个基因,基因与基因之间具有直接的,间接的,上调的或者下调的关系,我们可以从该层面获取基因网络动力学的拓扑结构以及其动力学性质。在单细胞层面,我们需要考虑的是基因网络对于一个细胞的具体行为会有何种程度的定量的影响。在多细胞层面,由于细胞之间的细胞因子通讯,我们需要考虑细胞与细胞之间的相互作用,而这又将是一个多体问题,而目前并没有一个显著有效的数学框架来进行描述。对于癌症这样的复杂过程的定量描述依然还是空白,有各种不同尺度的数学模型(时滞的微分方程或偏微分方程组)被提出来进行一些简单的刻画。但是如果我们在每个层面都能进行有效的分析与粗粒化,并且对多层次,多尺度进行有效的分析与描述,或许在不久的未来,我们可以为癌症的发生发展提出足够精确的定量描述与预测。


对于如此复杂的生物多体系统,想要进行有效的描述并不是一件容易的事,我们十分的关注一个复杂系统长时间的行为。在数学上,理论工具主要为非线性动力学,这是一个定量及定性刻画的有效工具。如果我们将复杂系统看作一个n维的高维体系,那么它的轨迹将处在一个n维的流形上,可以称之为生物体系的伪轨迹。我们需要考虑的是整个伪轨迹的演化,以及当时间趋于无穷时,伪轨迹的极限集的构成情况。在数学上,非线性动力系统有时具有极大的结构不稳定性,当某些个别参数发生变化,可能导致体系的极限集发生明显的拓扑学改变,称之为动力系统的分岔现象。目前对于高余维的分岔的了解我们还知之甚少,甚至是对于二维及以上情况也是如此。分岔会使得流形上的伪轨迹出现不规则的运行,产生极其复杂的变化。然而,在复杂体系的描述过程中,分岔理论是必不可少的。这其中有基于分岔理论的复杂疾病临界理论\cite{1},或是基于中心流形定理而发现的稳定的生物网络的核心结构\cite{2}。目前正在有许多基于动力系统的理论而发展出的定量生物学模型。


如果想要从复杂系统中看出其具体的演化规律,我们需要将非线性动力系统理论与微分几何相结合,通过将n维的伪轨迹进行降维到一个低维度的中心流形上。并且我们可以进一步的分析在低维流形上该轨迹的几何特征,例如生物演化过程对应的伪轨迹与该流形上的测地线的比较,是否其具有相一致性。在该中心流形上,我们能否找到最优的生物学演化路径,这对于生物学与数学都是极其重要的。定量生物学还有很多未知的问题,不论是理论上,实验上或是计算上,该领域目前只展露出冰山一角,有巨大的前景与有趣的理论等待我们去探索。
