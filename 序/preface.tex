\documentclass[12pt]{article}
\usepackage[UTF8]{ctex}
\usepackage{CJK}

\begin{document}
\begin{center}
   \Huge 写在前面的
\end{center}


我一直觉得,致远人需要一个学术期刊。

本人于2016年进入致远学院。大一的时候我深切地感到学院为了激发我们对科学的热情,拓宽我们对学术的视野下足了功夫。除了以科研实践为导向的课程设置外,我们听讲座,听沙龙,听报告,上研讨课,与大师面对面,听小故事,甚至很早就有机会进实验室。去年开始又有致远学者项目,致远学术节等等。

然而两年以来,总是觉得听过的名词术语多,真正深入了解的东西少;参与的科研实践多,内心的成就感少;关注和交流的内容以成绩和利益为多,而以纯粹的知识和学术为少。热情被现实浇灭的多,被激发的少。

有同学调侃过:致远现在有点留美预备学校的味道。

物理方向有门课程叫科研实践,每两周要写四页的科研实践报告。结果是每个双周周末热心的班长在班群里提醒各位“科研实践警报”,我们互相交流如何凑四页纸出来。唯一支持我通宵打字的动力,大概就是那微不足道的一学分。

略显荒唐的是,实践报告只是给老师看的,同学之间如果不主动询问竟然互相不知道别人对什么感兴趣,在做什么研究。

于是内心开始有这样的想法:要是有一个像知乎一样分享和交流的平台就好了,要是大家能不那么功利就好了。
如果现在努力的目标是仅仅是拿好的分数申请好的学校,做科研的时候写文章仅仅是为了评职称申经费找教职,那学术还有何乐趣?
似以为,学术与科研的乐趣在于永无止境的探索,学习,与创作。伟大的科学家往往是高产的作家。

所以,重新创办人刊的初衷就是鼓励同学们进行与学术相关的创作。内容可以是简单的对课程的总结,对你自学过的教材的介绍,对某个习题的深入探索,对一个领域的调研与介绍,也可以是科研实践的收获与成果,甚至可以是你对当代科学领域的批判性思考,对科学史的思考。

只要你有一定水平且愿意分享,这里就是你的舞台。

或许,在这里进行学术写作的价值不仅仅在于获得别人的认可,更重要的意义是锻炼自己学术思考的能力,总结知识的能力,融汇贯通的能力。

另外,希望人刊能成为一个传承思想的平台。

每一位低年级的致远学子想必都会有这些困惑:某某领域究竟在研究什么?难点到底在哪里?本科知识是如何运用在科研当中的?某篇论文,某个领域的某个突破价值究竟在哪里?科研的环境究竟是怎样的?有哪些研究领域比较有前景?为什么?

致远人早已遍及世界各地的各个科研领域。其实有的时候一篇文章就可以解决你很多困惑。如果人刊能起到一个传承的作用,学长学姐愿意分享这些正宗的学术干货,那么同学们一定有更多机会找准自己的定位与爱好。

希望致远学子在今后能踊跃投稿!

未来或许在某领域大奖的获奖感言里能加上一句:
\begin{center}
  \textbf
  “我人生中的第一篇学术论文发表于致远人刊!”
\end{center}

叶卓杨 $2019$年$1$月$17$日晚$9$点$36$分于西$13$寝室

\end{document}
