\PaperTitle{核磁共振陀螺仪(Nuclear Magnetic Resonance Gyroscope)} % Article title


\Authors{曹铭耘\textsuperscript{1}*} % Authors
\affiliation{
	\quad
	\textsuperscript{1}\textit{上海交通大学致远学院}
	\qquad
	\textsuperscript{2}\textit{上海交通大学自然科学研究院}
	\qquad
	*\textbf{邮箱地址}: caomingyun@sjtu.edu.cn
} % Author affiliation

\Abstract{
	\phantom{田田}时至今日,惯性导航技术受到了越来越多的重视。陀螺仪作为惯性导航系统的核心元件,相关的开发工作始终没有停止。核磁共振陀螺仪作为一种新型的陀螺仪,具有其它种类陀螺仪无法比拟的优势。其较小的体积使得将惯性导航装备于小型装置成为可能。长久以来,核磁共振陀螺仪受制于稳恒弱静磁场的技术瓶颈,实用价值较低。直到科学家提出使用两种元素来消除对静磁场的依赖,才使得核磁共振陀螺仪的实用化成为可能。
}


\Keywords{\phantom{田田}惯性导航 \quad 陀螺仪 \quad 核磁共振 \quad 两种元素 } % 如不需要关键词可直接删去花括号中内容
